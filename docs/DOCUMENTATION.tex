\documentclass[11pt]{article}
\usepackage{listings}
\usepackage{mathtools}
\usepackage{multicol}
\usepackage{hyperref}
\hypersetup{
  colorlinks   = true, %Colours links instead of ugly boxes
  urlcolor     = blue, %Colour for external hyperlinks
  linkcolor    = black, %Colour of internal links
  citecolor    = red %Colour of citations
}
% \usepackage{go}
% \usepackage{psgo}

    \title{\textbf{goma}}
    \author{gsobell}
    \date{last updated: \today}
    \addtolength{\topmargin}{-3cm}
    \addtolength{\textheight}{3cm}
\begin{document}
\maketitle
\tableofcontents
\newpage

\pagenumbering{arabic}

\subsection{Preface}
\href{https://github.com/gsobell/goma}{\textbf{goma}} is a go engine, and it's scope and functionality is therefore limited to being an engine. For a go controller, see \href{https://github.com/gsobell/dango}{\textbf{dango}}. For clarity, all mention of groups will refer to chains, that is, groups that are connected.

\section{GTP Implementation and Conformation}
According to the Go Text Protocol (GTP) documentation:
An engine is expected to keep track of the following state information:

\begin{itemize}
\item board size
\item board configuration
\item number of captured stones of either color
\item move history
\item komi
\item time settings
\end{itemize}
All implementations are required to support the following commands:

\begin{multicols}{2}
\begin{itemize}
\item protocol\_version
\item name
\item version
\item known\_command
\item list\_commands
\item quit
\item boardsize
\item clear\_board
\item komi
\item play
\item genmove
\end{itemize}
\end{multicols}

\section{State}
The stones are stored both as a 2D array, and as a list of lists, one for each color. Each list in the list is a group.
Black is -1, white is 1, empty spot is 0.
\subsection{Board}
The board is stored as a 2D array.
\subsection{Stones}
\subsubsection{Groups}
\subsubsection{Empty}

\section{Logic}
\subsection{Weighing Move Priority}
\subsection{Capture}
\subsection{Defense}
\subsection{Heuristics}

\section{Additional Resources}

\end{document}


